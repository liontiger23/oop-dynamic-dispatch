%\usepackage{xcolor}   % for \textcolor

%% Fix for "Missing number, treated as zero spaces"
%% when using consecutive spaces
\newsavebox\grayarrow
\sbox\grayarrow{\raisebox{0ex}[0ex][0ex]{\ensuremath{\textcolor{CtpOverlay1}\hookrightarrow\space}}}


\lstdefinestyle{default}{
    % Note: \small\color{CtpText} in basicstyle interferes with style in inlined text
    basicstyle   = {\ttfamily},
    stringstyle  = {\color{CtpGreen}},
    commentstyle = {\color{CtpOverlay1}},
    keywordstyle = {\color{CtpMauve}},
    keywordstyle = [2]{\color{CtpBlue}},
    keywordstyle = [3]{\color{CtpYellow}},
    keywordstyle = [4]{\color{CtpLavender}},
    keywordstyle = [5]{\color{CtpPeach}},
    keywordstyle = [6]{\color{CtpTeal}},
    captionpos   = b,
    breaklines   = false,
    numberstyle  = \tiny\color{CtpOverlay1},
    numbersep    = 5pt,
    escapeinside = {\$\$},
    columns      = flexible,
    postbreak    = \usebox\grayarrow,
    showstringspaces = false,
    texcl        = true,
}

\lstdefinestyle{bash}{
    style = default,
    language  = bash,
    emph      = [0]{local},
    emphstyle = [0]{},
}

\lstdefinestyle{diff}{
  style=default,
  morecomment=[n][\color{CtpTeal}]{@@}{@@\^^M},     % group identifier
  morecomment=[n][\color{CtpRed}]{-}{\^^M},         % deleted lines 
  morecomment=[n][\color{CtpGreen}]{+}{\^^M},       % added lines
  morecomment=[n][\color{CtpOverlay1}]{---}{\^^M}, % Diff lheader lines (must appear after +,-)
  morecomment=[n][\color{CtpOverlay1}]{+++}{\^^M},
}

\lstset{style=default}

%%%%%%%
%% Allow skipping line numbers -- https://tex.stackexchange.com/a/215752
\let\origthelstnumber\thelstnumber
\makeatletter
\newcommand*\Suppressnumber{%
  \lst@AddToHook{OnNewLine}{%
    \let\thelstnumber\relax%
     \advance\c@lstnumber-\@ne\relax%
    }%
}

\newcommand*\Reactivatenumber[1]{%
  \lst@AddToHook{OnNewLine}{%
   \let\thelstnumber\origthelstnumber%
   %\setcounter{lstnumber}{\numexpr#1-1\relax}%
   \advance\c@lstnumber\@ne\relax%
  }%
}

\makeatother
%%%%%%%

\usepackage{mdframed}

%% Provides BVerbatim environment which boxes verbatim
%% so it can be centered -- https://tex.stackexchange.com/a/122197
\usepackage{fancyvrb}

%% Set numbered footnotes in minipages
\renewcommand{\thempfootnote}{\arabic{mpfootnote}}

%% Support for QR codes
\usepackage[nolinks]{qrcode}

%% Do not number tables
%\captionsetup[table]{labelformat=empty}

%% Enable strikethrough syntax \st of newer pandoc
\usepackage{soul}

\usepackage{csquotes}

%% Normal subscripts in listings using \textsubscript{i} -- https://tex.stackexchange.com/questions/63845/boldface-and-subscripts-in-verbatim-mode    
\usepackage{fixltx2e}

%% Hack for using latex envs inside markdown -- https://github.com/jgm/pandoc/issues/3145#issuecomment-302787889
\newcommand{\hideFromPandoc}[1]{#1}
\hideFromPandoc{
  \let\Begin\begin
  \let\End\end
}

%% TODO command
\newcommand\todo[1]{\textcolor{red}{#1}}

%%%%%%%%%%%%%%%%%%%
%% Beamer stuff
%%%%%%%%%%%%%%%%%%%

%% Change template for page number to be just current page number
%% and increase vertical offset
\setbeamertemplate{footline}{
  \hfill%
  \usebeamercolor[fg]{page number in head/foot}%
  \usebeamerfont{page number in head/foot}%
  \setbeamertemplate{page number in head/foot}[pagenumber]%
  \usebeamertemplate*{page number in head/foot}\kern1em\vskip1em%
}

%% Update size and color of page numbering
\setbeamerfont{page number in head/foot}{size=\footnotesize}
\setbeamercolor{page number in head/foot}{fg=main}

\setbeamercolor{normal text}{fg=main}
\setbeamercolor{item}{fg=main}
\setbeamercolor{background canvas}{bg=bgmain}

\setbeamercolor*{palette primary}{fg=maininverted,bg=bginverted}

\setbeamercolor{normal font}{parent=background canvas}
\AtBeginDocument{\usebeamercolor{normal font}}

%\setbeamercolor*{titlelike}{fg=main}
\setbeamercolor*{titlelike}{parent=palette primary}
\setbeamercolor{frametitle}{parent=palette primary}
\setbeamercolor{structer}{fg=main}
\setbeamercolor{block title}{fg=main}

%% Center block titles
%\setbeamerfont{block title}{size=\centering\large}

\setbeamertemplate{itemize items}[circle]
\setbeamercolor{itemize item}{fg=main}
\setbeamercolor{itemize subitem}{fg=main}
\setbeamercolor{enumerate item}{fg=main}
\setbeamercolor{enumerate subitem}{fg=main}

\setbeamersize{text margin left=1em,text margin right=1em}

\setbeamercolor{alerted text}{fg=googlered}

%% Remove "Figure X:" prefix of figures
\setbeamertemplate{caption}{\raggedright\insertcaption\par}

%% Hack for onlytextwidth in beamer columns -- https://github.com/jgm/pandoc/issues/4150#issuecomment-598041677
%% configure columns environment to use totalwidth=\textwidth only
\let\origcolumns\columns
\let\endorigcolumns\endcolumns
\renewenvironment{columns}[1][]{\origcolumns[onlytextwidth,#1]}{\endorigcolumns}

%% fix some issues with pandoc -- https://tex.stackexchange.com/a/426090
\makeatletter
\let\@@magyar@captionfix\relax
\makeatother

%%%%%%%%%%%%%%%%%%%
%% Tikz stuff
%%%%%%%%%%%%%%%%%%%

\usepackage{tikz}
\usetikzlibrary{shapes,matrix,calc,positioning,fit,graphs,arrows.meta,backgrounds,decorations.pathreplacing,overlay-beamer-styles}

\tikzset{>=latex}

%% Align text vertically with subscript
%\tikzset{text depth=.25ex}
%% Align text vertically without subscript
\tikzset{text depth=0}


%% Scope prefix support -- https://tex.stackexchange.com/a/128079
\makeatletter
\tikzset{%
  prefix/.code={%
    \tikzset{%
      name/.code={\edef\tikz@fig@name{#1 ##1}}
    }%
  }%
}
\makeatother

%%%%%%%%%%%%%%%
%% Tikz macro
%%%%%%%%%%%%%%%

\definecolor{dark}{RGB}{48,48,48}

%% Align text vertically with subscript
%\tikzset{text depth=.25ex}
%% Align text vertically without subscript
\tikzset{text depth=0}

\tikzset{node/.style={minimum width=2.5em, minimum height=2.5em,draw,font=\ttfamily}}
\tikzset{class/.style={node,rectangle}}
\tikzset{interface/.style={node,circle}}

\def\structnodewidth{4.8em}
\def\structnodeheight{4.3ex}

\tikzset{struct base/.style={minimum width=\structnodewidth, minimum height=\structnodeheight, inner sep=0}}
%\tikzset{head/.style={struct base,font=\small\ttfamily\bfseries,fill=gray!20}}
\tikzset{head/.style={struct base,font=\footnotesize\ttfamily\bfseries,fill=gray!20}}
%\tikzset{field/.style={struct base,font=\small\ttfamily}}
\tikzset{field/.style={struct base,font=\footnotesize\ttfamily}}

\tikzset{muted/.style={font=\scriptsize\ttfamily,text=gray,color=gray}}



\newenvironment{struct}[2][]{
    \begin{scope}[node distance=0]

    \coordinate [#1] (#2) {};
    \def\headnode{#2}
    \def\lastnode{#2}

    \newcommand{\header}[2]{
        \node [head] (##1) [below=of \lastnode] {##2};
        \draw (##1.south west) -- (##1.south east);
        \def\lastnode{##1}
    }

    \newcommand{\field}[4][draw=none]{
        \node [field] (##3) [below=of \lastnode] {##4};
        \draw [##1] (##3.south west) -- (##3.south east);
        \node [muted,left=0.5em of ##3,fill=white,inner sep=0] (##3 @extra) {##2};
        \def\lastnode{##3}
    }
} {
    \node [draw,inner sep=0,fit=(\headnode) (\lastnode)] {};
    \end{scope}
}

\newcommand{\connect}[4][0.5]{
    \draw [#4] [->] (#2) -| ($ (#2) !#1! (#3) $) |- (#3);
}

\newcommand{\imtR}[3]{
    \draw [decorate,decoration={brace,amplitude=0.5em}] (#1.north east) -- (#2.south east) node [midway,font=\footnotesize\ttfamily\bfseries,right=0.25em] {#3};
}

\newcommand{\imtL}[3]{
    \draw [decorate,decoration={brace,amplitude=0.5em,mirror}] (#1.north west) -- (#2.south west) node [midway,font=\footnotesize\ttfamily\bfseries,left=0.25em] {#3};
}

\def\attrsize{\footnotesize}

\newcommand{\attr}[2]{
    \draw [] (#1.north west) -- (#1.south west) node [minimum height=\structnodeheight,font=\attrsize\ttfamily,midway,left=0.25em] (#1 attr) {#2};
}


%%%%%%%%%%%%%%%%%%%
%% pgfplots stuff
%%%%%%%%%%%%%%%%%%%

\usepackage{pgfplots}
\usepackage{pgfplotstable}

\definecolor{googleblue}{HTML}{4285F4}
\definecolor{googlered}{HTML}{EA4335}
\definecolor{googleyellow}{HTML}{FBBC04}
\definecolor{googlegreen}{HTML}{34A753}
\definecolor{googleorange}{HTML}{FE6D00}

